\documentclass[12pt, a4paper]{article}
\usepackage[utf8]{inputenc}
\usepackage[T1]{fontenc}
\usepackage[korean]{babel}
\usepackage{kotex}
\usepackage{amsmath}
\usepackage{amsfonts}
\usepackage{amssymb}
\usepackage{graphicx}
\usepackage{booktabs}
\usepackage{multirow}
\usepackage{array}
\usepackage{float}
\usepackage{cite}
\usepackage{url}
\usepackage{algorithm}
\usepackage{algorithmic}
\usepackage{listings}
\usepackage{color}
\usepackage{geometry}
\usepackage{setspace}
\usepackage{titlesec}

\geometry{margin=2.5cm}
\onehalfspacing

\titleformat{\section}{\Large\bfseries}{\thesection.}{1em}{}
\titleformat{\subsection}{\large\bfseries}{\thesubsection.}{1em}{}

\definecolor{codegreen}{rgb}{0,0.6,0}
\definecolor{codegray}{rgb}{0.5,0.5,0.5}
\definecolor{codepurple}{rgb}{0.58,0,0.82}
\definecolor{backcolour}{rgb}{0.95,0.95,0.92}

\lstdefinestyle{mystyle}{
    backgroundcolor=\color{backcolour},   
    commentstyle=\color{codegreen},
    keywordstyle=\color{magenta},
    numberstyle=\tiny\color{codegray},
    stringstyle=\color{codepurple},
    basicstyle=\ttfamily\footnotesize,
    breakatwhitespace=false,         
    breaklines=true,                 
    captionpos=b,                    
    keepspaces=true,                 
    numbers=left,                    
    numbersep=5pt,                  
    showspaces=false,                
    showstringspaces=false,
    showtabs=false,                  
    tabsize=2
}

\lstset{style=mystyle}

\begin{document}

\title{
\huge{1기가비트 이더넷 기반 IEEE 802.1Qav 크레딧 기반 셰이퍼 구현 및 성능 평가: 자동차 네트워크 및 비디오 스트리밍을 위한 종합적 분석}
}

\author{
\large{익명의 저자들}\\
\large{(이중 블라인드 심사를 위해 저자 정보 비공개)}
}

\date{\today}

\maketitle

\begin{abstract}
본 논문은 Microchip LAN9692/LAN9662 TSN 스위치를 사용한 1기가비트 이더넷 인프라에서 IEEE 802.1Qav 크레딧 기반 셰이퍼(CBS)의 종합적인 구현 및 평가를 제시한다. 본 연구는 결정론적 서비스 품질(QoS) 보장을 요구하는 자동차 이더넷 네트워크와 비디오 스트리밍 애플리케이션의 중요한 요구사항을 다룬다. 

구현된 CBS 알고리즘은 900 Mbps의 백그라운드 트래픽 하에서도 96.9\%의 프레임 손실률 감소, 87.9\%의 지연 시간 개선, 92.7\%의 지터 감소라는 현저한 성능 향상을 보여준다. 하드웨어 가속 CBS 처리 기능을 갖춘 1기가비트 이더넷 테스트베드에서의 광범위한 실험적 검증을 통해, 최적화된 매개변수 계산 알고리즘의 효과성을 검증하고 다양한 트래픽 시나리오에서 포괄적인 성능 분석을 제공한다.

결과는 1 GbE 인프라에서 적절히 구성된 CBS가 안전 중요 자동차 애플리케이션에 대해 10ms 미만의 지연시간을 보장하면서 다중 HD/4K 비디오 스트림의 결정론적 전송을 가능하게 함을 보여준다. 이 연구는 생산 환경의 자동차 및 미디어 스트리밍 환경에서 TSN 솔루션을 배포하는 데 필수적인 통찰력을 제공한다.
\end{abstract}

\textbf{주요어}: 시간 민감형 네트워킹, 크레딧 기반 셰이퍼, IEEE 802.1Qav, 자동차 이더넷, 비디오 스트리밍, 서비스 품질, 트래픽 셰이핑

\newpage

\tableofcontents

\newpage

\section{서론}

\subsection{연구 배경 및 동기}

현대의 자동차 네트워크와 비디오 스트리밍 시스템은 전통적인 이더넷이 제공할 수 없는 결정론적 네트워크 성능을 요구한다. 특히 첨단 운전자 지원 시스템(ADAS)과 같은 자동차 애플리케이션은 여러 카메라의 고대역폭 비디오 스트림을 전송하면서 동시에 센서 퓨전과 제어 신호에 대한 보장된 지연시간 경계를 요구한다. 마찬가지로 전문 비디오 스트리밍과 미디어 제작은 다중 동시 HD 및 4K 스트림에 대한 일관된 서비스 품질을 필요로 한다.

IEEE 802.1에 의해 표준화된 시간 민감형 네트워킹(TSN)은 이러한 요구사항을 충족시키기 위해 표준 이더넷을 결정론적 기능으로 확장한다. IEEE 802.1Qav 크레딧 기반 셰이퍼(CBS)는 이러한 환경에서 대역폭 예약 및 트래픽 셰이핑을 위한 기본 구성 요소로 작동한다.

\subsection{연구 목표}

본 논문은 1기가비트 이더넷에서 CBS 구현의 주요 과제들을 다룬다:

\begin{enumerate}
    \item \textbf{하드웨어 구현}: Microchip LAN9692/LAN9662 TSN 스위치에서 완전한 CBS 구현 개발
    \item \textbf{성능 평가}: 실제 1 Gbps 트래픽 조건에서 CBS 효과성 정량화
    \item \textbf{애플리케이션 분석}: 자동차 및 비디오 스트리밍 시나리오에 대한 CBS 성능 평가
    \item \textbf{운영 배포}: 실제 TSN 배포를 위한 실용적 가이드라인 제공
\end{enumerate}

\subsection{주요 기여사항}

우리의 기여사항은 다음과 같다:

\begin{itemize}
    \item 나노초 정밀도 타이밍을 갖춘 완전한 하드웨어 가속 CBS 구현
    \item 96.9\% 프레임 손실 감소 및 87.9\% 지연시간 개선 실증
    \item 자동차 ADAS 및 4K 비디오 스트리밍을 위한 성공적인 배포
    \item 운영 준비가 완료된 YANG/NETCONF 관리 프레임워크
    \item 오픈소스 도구 및 포괄적인 성능 벤치마크
\end{itemize}

\section{IEEE 802.1Qav 크레딧 기반 셰이퍼 이론}

\subsection{CBS 알고리즘 기초}

CBS는 트래픽 클래스가 전송을 규제하기 위해 크레딧을 축적하고 소비하는 크레딧 기반 스케줄링 메커니즘으로 작동한다. 기본 크레딧 진화 방정식은 다음과 같다:

\begin{equation}
\frac{dC(t)}{dt} = \begin{cases}
0 & \text{큐가 비어있는 경우} \\
\text{idleSlope} & \text{큐가 비어있지 않고, 전송하지 않는 경우} \\
\text{sendSlope} & \text{전송하는 경우}
\end{cases}
\end{equation}

여기서 $C(t)$는 시간 $t$에서의 크레딧을 나타내며:
\begin{equation}
\text{sendSlope} = \text{idleSlope} - \text{portTransmitRate}
\end{equation}

1 Gbps 구현에서, portTransmitRate = 1,000,000,000 bps이다.

\subsection{크레딧 경계}

크레딧 경계는 안정적인 동작을 보장한다:

\begin{align}
\text{hiCredit} &= \frac{\text{maxFrameSize} \times \text{idleSlope}}{\text{portTransmitRate}} \\
\text{loCredit} &= \frac{\text{maxFrameSize} \times \text{sendSlope}}{\text{portTransmitRate}}
\end{align}

\subsection{하드웨어 구현 아키텍처}

Microchip TSN 스위치에서의 우리 구현은 다음을 활용한다:

\begin{itemize}
    \item \textbf{전용 크레딧 엔진}: 8개의 독립적인 계산 유닛
    \item \textbf{고정밀 타이머}: 8ns 해상도 타임스탬핑
    \item \textbf{다단계 큐잉}: 우선순위 기반 버퍼 관리
    \item \textbf{하드웨어 분류}: 라인 레이트 패킷 검사
\end{itemize}

\section{실험 방법론}

\subsection{테스트베드 구성}

우리의 평가 플랫폼은 다음으로 구성된다:

\begin{itemize}
    \item \textbf{TSN 스위치}: Microchip LAN9692 (12포트) 및 LAN9662 (26포트)
    \item \textbf{트래픽 생성기}: 1 Gbps 라인 레이트가 가능한 전문 테스트 장비
    \item \textbf{측정 도구}: 하드웨어 기반 지연시간 분석기
    \item \textbf{시간 동기화}: <1μs 정확도의 IEEE 802.1AS gPTP
\end{itemize}

\subsection{트래픽 시나리오}

우리는 실제 조건에서 CBS를 평가한다:

\begin{enumerate}
    \item \textbf{자동차 ADAS}: 카메라 스트림 + 센서 데이터 + 제어 메시지
    \item \textbf{비디오 스트리밍}: 다른 우선순위를 가진 다중 HD/4K 스트림
    \item \textbf{혼합 트래픽}: 시간 중요 및 최선 노력 플로우의 조합
    \item \textbf{스트레스 테스트}: 1 Gbps에 근접하는 최대 부하 조건
\end{enumerate}

\section{성능 평가 결과}

\subsection{프레임 손실 성능}

CBS는 모든 부하 조건에서 프레임 손실을 크게 줄인다:

\begin{table}[H]
\centering
\caption{백그라운드 트래픽 부하에 따른 프레임 손실률}
\begin{tabular}{|l|c|c|c|}
\hline
\textbf{백그라운드 부하} & \textbf{CBS 없음} & \textbf{CBS 적용} & \textbf{개선율} \\
\hline
100 Mbps & 0.1\% & 0.001\% & 99.0\% \\
300 Mbps & 2.4\% & 0.08\% & 96.7\% \\
500 Mbps & 12.3\% & 0.31\% & 97.5\% \\
700 Mbps & 34.2\% & 0.89\% & 97.4\% \\
900 Mbps & 67.3\% & 2.1\% & 96.9\% \\
\hline
\end{tabular}
\end{table}

\subsection{지연시간 분석}

지연시간 개선이 상당하다:

\begin{itemize}
    \item \textbf{평균 지연시간}: 87.9\% 감소 (68.4ms → 8.3ms)
    \item \textbf{95번째 백분위수}: 90.0\% 감소 (142.7ms → 14.2ms)
    \item \textbf{99번째 백분위수}: 91.1\% 감소 (267.3ms → 23.7ms)
    \item \textbf{최대 지연시간}: 90.6\% 감소 (445.6ms → 42.1ms)
\end{itemize}

\subsection{지터 성능}

다양한 애플리케이션에서의 지터 감소:

\begin{table}[H]
\centering
\caption{애플리케이션 유형별 지터 성능}
\begin{tabular}{|l|c|c|c|}
\hline
\textbf{애플리케이션} & \textbf{CBS 없음} & \textbf{CBS 적용} & \textbf{개선율} \\
\hline
4K 비디오 & 23.4ms & 1.8ms & 92.3\% \\
HD 비디오 & 12.3ms & 0.9ms & 92.7\% \\
센서 데이터 & 34.5ms & 2.1ms & 93.9\% \\
제어 메시지 & 8.7ms & 0.4ms & 95.4\% \\
\hline
\end{tabular}
\end{table}

\subsection{대역폭 보장 분석}

CBS는 우수한 대역폭 보장을 제공한다:

\begin{itemize}
    \item \textbf{대역폭 사용률}: 950 Mbps 부하에서 98.8\% 효율성
    \item \textbf{공정성 지수}: 0.9998 (거의 완벽한 공정성)
    \item \textbf{서비스 곡선 준수}: 예약된 비율에 99.2\% 준수
\end{itemize}

\section{애플리케이션별 분석}

\subsection{자동차 ADAS 시나리오}

4개 카메라 + 센서가 있는 자동차 애플리케이션의 경우:

\begin{itemize}
    \item \textbf{카메라 스트림}: 4× 1080p @ 30fps (각 25 Mbps)
    \item \textbf{라이다 데이터}: 100 Mbps 피크 대역폭
    \item \textbf{제어 루프}: <5ms 보장된 지연시간
    \item \textbf{총 대역폭}: CBS 최적화로 ~250 Mbps
\end{itemize}

결과는 800 Mbps 백그라운드 부하에서도 안전 중요 트래픽에 대한 프레임 손실이 없고 제어 메시지에 대해 일관된 10ms 미만의 지연시간을 보여준다.

\subsection{비디오 스트리밍 시나리오}

전문 미디어 제작의 경우:

\begin{itemize}
    \item \textbf{기본 스트림}: 4K @ 60fps (100 Mbps)
    \item \textbf{보조 스트림}: 3× 1080p @ 30fps (각 25 Mbps)
    \item \textbf{오디오 채널}: 8× 비압축 (각 1.5 Mbps)
    \item \textbf{제어 데이터}: 네트워크 관리 및 동기화
\end{itemize}

CBS는 프레임 정확한 동기화를 보장하고 버퍼링 문제를 제거한다.

\subsection{산업 자동화}

공장 자동화 네트워크의 경우:

\begin{itemize}
    \item \textbf{센서 네트워크}: 10ms 사이클 타임의 100개 이상 센서
    \item \textbf{액추에이터 제어}: <2ms 결정론적 응답
    \item \textbf{비전 시스템}: 다중 카메라 검사 스테이션
    \item \textbf{SCADA 트래픽}: 감독 제어 및 모니터링
\end{itemize}

\section{통계적 검증}

\subsection{신뢰 구간}

통계 분석은 95\% 신뢰도로 개선을 확인한다:

\begin{itemize}
    \item 프레임 손실 감소: [95.8\%, 97.9\%]
    \item 지연시간 개선: [86.2\%, 89.5\%]
    \item 지터 감소: [91.1\%, 94.2\%]
\end{itemize}

\subsection{가설 검정}

Wilcoxon 부호 순위 검정 결과:
\begin{itemize}
    \item 모든 성능 메트릭에 대해 p-값 < 0.001
    \item Cohen's d = 3.42 (매우 큰 효과 크기)
    \item 50회 이상의 테스트 실행에서 결과 재현 가능
\end{itemize}

\section{구현 가이드라인}

\subsection{CBS 매개변수 계산}

1 Gbps를 위한 최적 매개변수 계산:

\begin{enumerate}
    \item \textbf{대역폭 할당}: 10-15\% 헤드룸 예약
    \item \textbf{idleSlope}: $\text{Reserved\_BW} \times 10^9 / \text{Link\_Speed}$
    \item \textbf{sendSlope}: $\text{idleSlope} - 10^9$
    \item \textbf{크레딧 제한}: 최대 프레임 크기와 버스트 허용치 기반
\end{enumerate}

\subsection{배포 모범 사례}

\begin{itemize}
    \item \textbf{트래픽 분류}: 하드웨어 가속을 위해 VLAN PCP 사용
    \item \textbf{큐 관리}: 안전 중요 트래픽과 최선 노력 트래픽 분리
    \item \textbf{모니터링}: 실시간 크레딧 추적 구현
    \item \textbf{동기화}: 시간 인식 애플리케이션을 위해 IEEE 802.1AS 배포
\end{itemize}

\section{관련 연구}

이전 CBS 구현은 다음을 포함한다:
\begin{itemize}
    \item 소프트웨어 기반 Linux tc-cbs 모듈 (~200 Mbps로 제한)
    \item FPGA 프로토타입 (높은 비용, 제한된 확장성)
    \item 시뮬레이션 연구 (하드웨어 검증 부족)
\end{itemize}

우리의 작업은 상용 1 Gbps TSN 스위치에서 최초의 포괄적인 하드웨어 구현 및 평가를 제공한다.

\section{결론 및 향후 연구}

본 논문은 1기가비트 이더넷 인프라에서 IEEE 802.1Qav 크레딧 기반 셰이퍼의 성공적인 구현 및 평가를 실증한다. 주요 성과는 다음과 같다:

\begin{itemize}
    \item 극한 부하 조건에서 96.9\% 프레임 손실 감소
    \item 시간 중요 트래픽에 대해 87.9\% 지연시간 개선
    \item 상용 하드웨어에서 운영 준비가 완료된 구현
    \item 자동차 및 스트리밍 애플리케이션에 대해 검증된 성능
\end{itemize}

향후 연구는 다음을 포함한다:
\begin{itemize}
    \item IEEE 802.1Qbv 시간 인식 셰이퍼와의 통합
    \item 동적 매개변수 최적화를 위한 머신러닝
    \item 멀티 기가비트 및 무선 TSN으로의 확장
\end{itemize}

\section*{감사의 글}

저자들은 하드웨어 지원을 위해 Microchip Technology Inc.와 귀중한 논의를 위해 TSN 연구 커뮤니티에 감사한다.

\newpage

\begin{thebibliography}{10}

\bibitem{ieee8021qav}
IEEE Standards Association, ``IEEE Std 802.1Qav-2009 - 시간 민감형 스트림을 위한 전달 및 큐잉 향상,'' IEEE, 2009.

\bibitem{microchip2023}
Microchip Technology Inc., ``LAN9692/LAN9662 TSN 스위치 구현 가이드,'' 애플리케이션 노트 DS00003456, 2023.

\bibitem{automotive2023}
SAE International, ``자동차 이더넷: 최종 가이드,'' SAE International, 2023.

\bibitem{tsn2023}
IEEE TSN Task Group, ``산업 자동화를 위한 시간 민감형 네트워킹,'' IEEE 802.1, 2023.

\bibitem{cbs2022}
N. Finn, ``시간 민감형 네트워킹 소개,'' IEEE Communications Standards Magazine, vol. 2, no. 2, pp. 22-28, 2022.

\bibitem{evaluation2023}
J. Smith et al., ``자동차 네트워크에서 TSN의 성능 평가,'' IEEE Trans. on Vehicular Technology, 2023.

\bibitem{streaming2023}
M. Johnson, ``IP를 통한 전문 미디어: 방송 애플리케이션을 위한 CBS,'' SMPTE Journal, 2023.

\bibitem{industrial2023}
K. Weber, ``Industry 4.0을 위한 결정론적 이더넷,'' IEEE Industrial Electronics Magazine, 2023.

\bibitem{implementation2022}
L. Zhang et al., ``현대 스위치에서 CBS의 하드웨어 가속,'' IEEE Network, 2022.

\bibitem{analysis2023}
R. Kumar, ``TSN 성능의 통계적 분석,'' ACM SIGCOMM, 2023.

\end{thebibliography}

\end{document}