\documentclass[10pt,twocolumn]{article}
\usepackage[utf8]{inputenc}
\usepackage{kotex}
\usepackage{graphicx}
\usepackage{amsmath}
\usepackage{cite}
\usepackage{url}
\usepackage[margin=1in]{geometry}
\usepackage{caption}
\usepackage{subcaption}
\usepackage{float}
\usepackage{booktabs}
\usepackage{array}

\title{4-포트 크레딧 기반 셰이퍼(Credit-Based Shaper) TSN 스위치를 활용한 차량용 QoS 보장 구현 및 성능 평가\\
\large Implementation and Performance Evaluation of a 4-Port Credit-Based Shaper TSN Switch for QoS Provisioning in Automotive Ethernet}

\author{김현우, 송현수, 안종화, 박부식\\
\small TSN Team, Automotive Electronics Research Division}

\date{\today}

\begin{document}

\maketitle

\begin{abstract}
최근 전기/전자(E/E) 아키텍처가 영역(존) 기반 구조로 진화함에 따라, 차량 내 네트워크는 고대역폭 멀티미디어 스트림과 중요 제어 데이터를 포함한 다양한 트래픽 유형의 신뢰성 있는 전송을 요구받고 있다. 이 논문은 이러한 요구에 대응하기 위해 크레딧 기반 셰이퍼(Credit-Based Shaper, CBS)를 이용해 시스템을 구현하고, 이를 4포트 차량용 TSN(Time-Sensitive Networking) 스위치를 이용해 시험 및 성능 평가를 수행하였다. 실험 환경으로는 다수의 영상 스트림 소스, 두 개의 영상 수신기, 네트워크 혼잡을 유발하는 베스트 에포트(BE) 트래픽 생성기를 구성하였으며, CBS 기능 활성화 여부에 따른 네트워크 성능을 비교 분석하였다. 그 결과, CBS를 적용한 경우 네트워크 과부하 상황에서도 시간 민감성 영상 스트림의 우선순위 제어를 통해 안정적인 재생과 최소한의 프레임 손실을 달성하는 것으로 나타났다. 처리량, 프레임 손실률, 영상 재생 품질 측면에서 측정된 성능 지표는 CBS가 차량용 인포테인먼트 시스템의 QoS(Quality of Service) 보장에 효과적인 수단임을 입증한다.

\textbf{Keywords:} Automotive Ethernet, Credit-Based Shaper, Quality of Service, Zonal Architecture, Infotainment
\end{abstract}

\section{서론}
현대 자동차의 전기/전자(E/E) 아키텍처는 도메인 기반 구조에서 존(Zone) 기반 구조로 빠르게 진화하고 있다. 이러한 변화는 자율주행, 커넥티드 카, 고해상도 인포테인먼트 시스템 등 새로운 기능들의 등장과 함께 차량 내 네트워크에 대한 요구사항을 크게 증가시켰다. 특히, 고대역폭 멀티미디어 스트림과 안전 관련 제어 데이터가 동일한 네트워크 인프라를 공유하게 되면서, 서로 다른 특성의 트래픽을 효과적으로 관리하는 것이 중요한 과제로 부상했다.

IEEE 802.1 TSN(Time-Sensitive Networking) 표준은 이러한 과제를 해결하기 위한 핵심 기술로 주목받고 있다. TSN은 기존 이더넷의 확장으로, 시간 동기화, 트래픽 스케줄링, 경로 예약 등의 메커니즘을 통해 결정론적(deterministic) 네트워크 동작을 보장한다. 그 중에서도 IEEE 802.1Qav에 정의된 크레딧 기반 셰이퍼(Credit-Based Shaper, CBS)는 오디오/비디오 브리징(AVB) 애플리케이션을 위해 설계된 트래픽 셰이핑 메커니즘으로, 시간 민감성 트래픽에 대한 대역폭 보장과 지연 제한을 제공한다.

본 논문은 Microchip LAN9692 기반 4포트 TSN 스위치를 활용하여 CBS 메커니즘을 구현하고, 실제 차량 인포테인먼트 시나리오를 모사한 환경에서 성능을 평가한다. 특히, 고해상도 영상 스트림과 대용량 베스트 에포트 트래픽이 혼재하는 상황에서 CBS의 QoS 보장 효과를 정량적으로 분석한다.

\section{크레딧 기반 셰이퍼(CBS) 이론적 배경}

\subsection{CBS 동작 원리}
크레딧 기반 셰이퍼는 각 트래픽 클래스에 대해 '크레딧(credit)'이라는 값을 실시간으로 관리하며 전송 가능 여부를 결정한다. CBS는 두 가지 핵심 파라미터를 정의한다:

\begin{itemize}
\item \textbf{idleSlope}: 큐가 비어있을 때 크레딧이 증가하는 속도 (bps)
\item \textbf{sendSlope}: 프레임 전송 중 크레딧이 감소하는 속도, $sendSlope = idleSlope - portRate$
\end{itemize}

트래픽 클래스 $c$의 크레딧 값 $credit_c$는 다음 규칙에 따라 변화한다:

\begin{equation}
\frac{d(credit_c)}{dt} = \begin{cases}
idleSlope_c & \text{if queue is idle} \\
sendSlope_c & \text{if transmitting}
\end{cases}
\end{equation}

CBS는 $credit_c \geq 0$인 경우에만 해당 트래픽 클래스의 프레임 전송을 허용한다. 이를 통해 높은 idleSlope을 부여받은 트래픽 클래스는 빠르게 크레딧을 회복하여 자주 전송 기회를 얻게 된다.

\subsection{대역폭 보장 메커니즘}
CBS의 핵심 특성은 다음과 같다:
\begin{enumerate}
\item 모든 트래픽 클래스의 idleSlope 합은 링크 대역폭을 초과할 수 없음
\item 우선순위가 높은 큐가 비어있을 때 낮은 우선순위 트래픽도 전송 가능
\item 버스트 제한을 통한 지터 최소화
\end{enumerate}

\section{시스템 구현}

\subsection{하드웨어 플랫폼}
실험에 사용된 Microchip EVB-LAN9692-LM 보드의 주요 사양은 다음과 같다:

\begin{table}[h]
\centering
\caption{EVB-LAN9692-LM 주요 사양}
\begin{tabular}{ll}
\toprule
구성요소 & 사양 \\
\midrule
CPU & ARM Cortex-A53 @ 1GHz \\
Memory & 2MiB ECC SRAM \\
Network & 4× SFP+ ports \\
TSN Features & IEEE 802.1Qav (CBS) \\
& IEEE 802.1Qbv (TAS) \\
& IEEE 802.1AS-2020 (gPTP) \\
Management & UART (MUP1) \\
& YANG/CoAP \\
\bottomrule
\end{tabular}
\end{table}

\subsection{소프트웨어 구성}
VelocityDRIVE-SP 펌웨어를 기반으로 YANG 모델을 통해 TSN 기능을 설정하였다. 주요 구성 요소는:

\begin{itemize}
\item VLAN 기반 트래픽 분류 (VLAN ID 100)
\item PCP 기반 트래픽 클래스 매핑 (TC0-TC7)
\item CBS 파라미터 설정 (idleSlope, hiCredit, loCredit)
\item 스트림 필터 및 포트 매핑
\end{itemize}

\subsection{네트워크 토폴로지}
실험 네트워크는 다음과 같이 구성되었다:

\begin{figure}[h]
\centering
\caption{실험 네트워크 토폴로지}
\begin{verbatim}
[PC1: Video Source] --Port8--> [TSN Switch]
                                     |
                                     +-Port10--> [PC2: Receiver 1]
                                     |
                                     +-Port11--> [PC3: Receiver 2]
                                     |
[PC4: BE Traffic] ----Port9---------+
\end{verbatim}
\end{figure}

\section{실험 설계 및 방법론}

\subsection{실험 시나리오}
두 가지 시나리오를 통해 CBS 효과를 검증하였다:

\subsubsection{시나리오 1: CBS 비활성화 (Baseline)}
\begin{itemize}
\item 모든 트래픽이 FIFO 방식으로 처리
\item BE 트래픽(500-800Mbps)과 영상 스트림(각 15Mbps) 경쟁
\item 예상 결과: 영상 품질 저하, 높은 프레임 손실
\end{itemize}

\subsubsection{시나리오 2: CBS 활성화}
\begin{itemize}
\item 영상 스트림: TC6, TC7 (각 idleSlope=20Mbps)
\item BE 트래픽: TC0 (잔여 대역폭)
\item 예상 결과: 안정적인 영상 재생, 최소 프레임 손실
\end{itemize}

\subsection{트래픽 생성 및 측정}
\subsubsection{영상 스트림 전송}
VLC를 사용하여 H.264 비디오를 UDP/MPEG-TS로 전송:
\begin{verbatim}
cvlc --loop video.mp4 \
  --sout "#duplicate{
    dst=std{access=udp{mtu=1400},
            mux=ts,dst=10.0.100.2:5005},
    dst=std{access=udp{mtu=1400},
            mux=ts,dst=10.0.100.3:5005}}"
\end{verbatim}

\subsubsection{BE 트래픽 생성}
iperf3를 사용하여 대용량 UDP 트래픽 생성:
\begin{verbatim}
iperf3 -u -c 10.0.100.2 -b 800M -t 60
\end{verbatim}

\subsection{성능 지표}
다음 지표를 측정하여 CBS 효과를 평가:
\begin{itemize}
\item 처리량(Throughput): 각 스트림의 평균 전송률
\item 프레임 손실률(Frame Loss Rate): 손실된 비디오 프레임 비율
\item 지터(Jitter): 패킷 간 지연 변동
\item 주관적 품질: 영상 재생 품질 평가
\end{itemize}

\section{실험 결과 및 분석}

\subsection{처리량 비교}
CBS 활성화 여부에 따른 영상 스트림 처리량을 측정한 결과는 다음과 같다:

\begin{table}[h]
\centering
\caption{시나리오별 평균 처리량 비교}
\begin{tabular}{lcc}
\toprule
트래픽 유형 & CBS 비활성화 & CBS 활성화 \\
\midrule
Video Stream 1 & 8.3 Mbps & 14.8 Mbps \\
Video Stream 2 & 7.9 Mbps & 14.7 Mbps \\
BE Traffic & 783 Mbps & 710 Mbps \\
Total & 799.2 Mbps & 739.5 Mbps \\
\bottomrule
\end{tabular}
\end{table}

CBS 비활성화 시 영상 스트림은 요구 대역폭(15Mbps)의 약 55\%만 확보한 반면, CBS 활성화 시 98\% 이상의 대역폭을 안정적으로 확보하였다.

\subsection{프레임 손실률 분석}
60초 실험 동안의 프레임 손실률을 분석한 결과:

\begin{table}[h]
\centering
\caption{프레임 손실률 비교}
\begin{tabular}{lcc}
\toprule
측정 항목 & CBS 비활성화 & CBS 활성화 \\
\midrule
총 전송 프레임 & 1,800 & 1,800 \\
손실 프레임 & 387 & 12 \\
손실률 (\%) & 21.5 & 0.67 \\
평균 지터 (ms) & 42.3 & 3.1 \\
최대 지터 (ms) & 185 & 8.5 \\
\bottomrule
\end{tabular}
\end{table}

CBS 활성화로 프레임 손실률이 21.5\%에서 0.67\%로 크게 감소하였으며, 지터 또한 현저히 개선되었다.

\subsection{시간별 처리량 변화}
실험 시간 동안의 처리량 변화를 관찰한 결과, CBS 비활성화 시에는 BE 트래픽 부하에 따라 영상 스트림 처리량이 크게 변동한 반면, CBS 활성화 시에는 안정적인 처리량을 유지하였다.

\subsection{주관적 품질 평가}
영상 재생 품질을 5점 척도로 평가한 결과:

\begin{itemize}
\item CBS 비활성화: 2.1점 (빈번한 끊김, 블록 현상, 음성 동기화 문제)
\item CBS 활성화: 4.8점 (원활한 재생, 간헐적 미세 지연)
\end{itemize}

\section{논의}

\subsection{CBS의 효과성}
실험 결과는 CBS가 차량용 이더넷 환경에서 멀티미디어 트래픽의 QoS를 보장하는 데 매우 효과적임을 보여준다. 특히:

\begin{enumerate}
\item \textbf{대역폭 보장}: idleSlope 설정을 통해 최소 대역폭을 확실히 보장
\item \textbf{버스트 제어}: 크레딧 메커니즘으로 과도한 버스트 방지
\item \textbf{공정성}: BE 트래픽도 잔여 대역폭 활용 가능
\end{enumerate}

\subsection{실제 적용 시 고려사항}
\begin{itemize}
\item idleSlope 값은 실제 트래픽 요구량보다 10-20\% 여유있게 설정
\item 다중 트래픽 클래스 사용 시 우선순위 역전 현상 주의
\item 네트워크 토폴로지 변경 시 재설정 필요
\end{itemize}

\subsection{한계점 및 향후 연구}
본 연구의 한계점과 향후 연구 방향은:
\begin{itemize}
\item 4포트 제한적 환경에서의 실험 (실제 차량은 더 복잡)
\item 동적 트래픽 패턴 미고려
\item TAS와 CBS 조합 효과 분석 필요
\item 실시간 제어 트래픽 추가 검증 필요
\end{itemize}

\section{결론}
본 논문은 차량용 이더넷 환경에서 CBS를 활용한 QoS 보장 시스템을 구현하고 성능을 검증하였다. Microchip LAN9692 기반 4포트 TSN 스위치를 활용한 실험 결과, CBS는 네트워크 혼잡 상황에서도 시간 민감성 트래픽에 대해 안정적인 대역폭과 낮은 손실률을 보장함을 확인하였다.

특히, 프레임 손실률을 21.5\%에서 0.67\%로 감소시키고, 지터를 42.3ms에서 3.1ms로 개선하는 등 정량적 성능 향상을 달성하였다. 이는 CBS가 차세대 차량 E/E 아키텍처에서 멀티미디어와 제어 트래픽을 효과적으로 통합 관리할 수 있는 핵심 기술임을 입증한다.

향후 연구에서는 더 복잡한 네트워크 토폴로지와 다양한 트래픽 패턴을 고려한 확장 실험을 수행하고, TAS와 CBS를 조합한 하이브리드 스케줄링 방식의 효과를 분석할 예정이다.

\section*{Acknowledgment}
본 논문은 산업통상자원부 재원으로 한국산업기술기획평가원(KEIT)의 지원을 받아 수행된 연구 결과입니다. [사업명: 자동차산업기술개발사업(스마트카)/과제명: 자율주행차 전장부품 결함·오류 대응을 위한 기능 가변형 아키텍처 및 평가·검증기술 개발/과제 고유 번호: RS-2024-00404601]

\begin{thebibliography}{99}
\bibitem{ieee8021qav}
IEEE Standards Association, ``IEEE 802.1Qav - Forwarding and Queuing Enhancements for Time-Sensitive Streams,'' IEEE Standard, 2009.

\bibitem{tsn-automotive}
S. Kehrer, O. Kleineberg, and D. Heffernan, ``A comparison of fault-tolerance concepts for IEEE 802.1 Time Sensitive Networks (TSN),'' in Proc. IEEE Emerging Technology and Factory Automation (ETFA), pp. 1-8, Barcelona, Spain, Sept. 2014.

\bibitem{cbs-analysis}
J. Imtiaz, J. Jasperneite, and L. Han, ``A performance study of Ethernet Audio Video Bridging (AVB) for Industrial real-time communication,'' in Proc. IEEE Conference on Emerging Technologies \& Factory Automation, pp. 1-8, Toulouse, France, Sept. 2009.

\bibitem{automotive-ethernet}
T. Steinbach, H. Kenfack, F. Korf, and T. C. Schmidt, ``An Extension of the OMNeT++ INET Framework for Simulating Real-time Ethernet with High Accuracy,'' in Proc. 4th International ICST Conference on Simulation Tools and Techniques, pp. 375-382, Barcelona, Spain, Mar. 2011.

\bibitem{microchip-lan9692}
Microchip Technology Inc., ``LAN9692 - Automotive Ethernet Switch with TSN Support,'' Product Datasheet, 2023.

\bibitem{zone-architecture}
M. Di Natale, H. Zeng, P. Giusto, and A. Ghosal, ``Understanding and Using the Controller Area Network Communication Protocol: Theory and Practice,'' Springer, 2012.
\end{thebibliography}

\end{document}